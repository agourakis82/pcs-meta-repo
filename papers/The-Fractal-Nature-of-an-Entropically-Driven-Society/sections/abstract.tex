\abstract{%
\textbf{Background:}  Macroscopic social phenomena—imperial cycles, market collapses, megacity growth—display the same scale‑invariant signatures that characterise branching lungs or river basins.  I advance the thesis that human collectives form a \emph{fractal, entropically‑driven system} whose visible patterns arise from hidden flows of configurational entropy.%

\textbf{Methods:}  Building on a symbolic Schrödinger equation and a social Hamiltonian defined in terms of ideological inertia, I derive a closed‑form fractal dimension \(D\) and validate it through Monte‑Carlo simulations across five hierarchical scales (family → civilisation) and two centuries of census data.%

\textbf{Results:}  The model predicts a universal exponent \(D = 1.58 \pm 0.03\) and identifies a bifurcation coefficient \(\mathcal{R}_{\mathrm{crit}}\!\approx\!1.7\) that precedes eighty‑four per cent of documented systemic reorganisations—including Roman, Tang and Ottoman transitions, and the crashes of 1929, 2000 and 2008 .  The analytical proof and numerical table are reproduced verbatim in Section 2.5.%

\textbf{Conclusions:}  Entropic constraints alone suffice to engender emergent social order, rendering agent‑specific heuristics unnecessary and unifying statistical physics, complexity sociology and philosophy of mind.  The framework offers quantitative early‑warning indicators for critical transitions and an ontological bridge between physical and symbolic domains.%
}
\keywords{entropy; fractal geometry; complex systems; social physics; symbolic wave‑function; self‑organisation
}
