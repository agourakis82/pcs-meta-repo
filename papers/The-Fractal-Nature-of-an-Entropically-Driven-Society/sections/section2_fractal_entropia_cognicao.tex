\section{The Fundamental Concept of Fractals}

Fractals are structures that defy classical perceptions of shape and proportion. Their defining feature is not geometric regularity, but dynamic self-similarity: organised patterns that repeat, with scalar variations, across multiple levels of complexity. In other words, the whole and the part share a common structural logic, as if the universe were written by a principle that folds recursively upon itself.

A branch mirrors the logic of a tree; a capillary repeats the hierarchy of systemic circulation; a local social pattern echoes, on an amplified scale, the configuration of civilisations. Fractal geometry is not static — it is alive, expansive, fluid. At each scale, the structure renews itself according to the same organising principles, making it paradoxically both simple and complex.

The modern formulation of the term was established by Benoît Mandelbrot, a Polish-born French mathematician, who coined it in 1975 to describe forms whose details recur indefinitely across scales, and which could not be described by classical Euclidean geometry. According to Mandelbrot, a fractal is a figure “whose Hausdorff dimension exceeds its topological dimension” and which “exhibits similarity across multiple scales” \cite{mandelbrot1982}.

In his most elegant and concise definition, Mandelbrot wrote:

\begin{quote}
“A fractal is, by definition, a set for which the Hausdorff–Besicovitch dimension strictly exceeds its topological dimension.” \cite{mandelbrot1982}
\end{quote}

This condition implies that fractals are not merely mathematical curiosities, but formal expressions of a kind of organised complexity that repeats itself. To understand the fractal nature of reality is to relinquish linearity in favour of a deeper logic: transformed repetition as the engine of complexity.
