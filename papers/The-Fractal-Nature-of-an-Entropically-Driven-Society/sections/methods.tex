%=================================================================
\section{Mathematical Framework and Simulation Protocol}
\label{sec:methods}
%=================================================================

\subsection{Symbolic Schrödinger formalism}

We model society as a wave‑mechanical system whose state is described by a
symbolic wave function $\psi(\mathbf{x},t)$, where
$\mathbf{x}=(x_{1},\ldots,x_{n})$ spans a multidimensional **symbolic
space**—affect, belonging, ideology, power, memory—and $t$ is historical
time \citep{mandelbrot1982,schrodinger1944}.
The governing equation is a socio‑symbolic analogue of the
Schrödinger equation :

\begin{equation}
\label{eq:social_schroedinger}
\mathrm{i}\hbar\frac{\partial\psi}{\partial t}
  =\Bigl[-\frac{\hbar^{2}}{2m_{\mathrm{s}}}\nabla^{2}+V(\mathbf{x})\Bigr]\psi,
\end{equation}

where $m_{\mathrm{s}}$ is the \emph{symbolic mass}—resistance to
ideological change—and $V(\mathbf{x})$ is the symbolic potential
landscape generated by norms, narratives, repression, and desire .

\subsection{Fractal bifurcation coefficient \texorpdfstring{$\mathcal{R}$}{R}}

Let $P(s,t)$ denote the aggregate symbolic momentum at scale
$s\!\in\!\{1,\ldots,S\}$ (family $\rightarrow$ civilisation).
Define \citep{sporns2011,scheffer2009}:

\begin{equation}
\label{eq:R_def}
\mathcal{R}(t)=\frac{1}{S\,\sigma_{P}(t)}
\sum_{s=1}^{S}\bigl|\partial_{t}P(s,t)\bigr|,
\qquad
\sigma_{P}^{2}(t)=\frac{1}{S}\sum_{s=1}^{S}
\Bigl(\partial_{t}P(s,t)-\mu(t)\Bigr)^{2},
\end{equation}

with $\mu(t)$ the mean temporal derivative across scales.
Section~2.5 of the Portuguese master text proves:

> **Theorem 1 (Fractal bifurcation)**
> If $\mathcal{R}(t)\ge\mathcal{R}_{\text{crit}}\!\approx\!1.7$ there
> exists at least one scale $s^{\ast}$ where
> $\partial_{t}P(s^{\ast},t)$ changes sign, triggering a structural
> bifurcation.

\paragraph{Sketch of proof.}
Applying Chebyshev’s inequality to the set
$\{x_{s}=\partial_{t}P(s,t)\}$ yields a lower bound on the maximal
deviation; for $k\!=\!\mathcal{R}(t)$,
$|x_{s^{\ast}}|\ge k\sigma_{P}$. When $k>1$
and the set contains mixed signs, an inversion of sign at some
$s^{\ast}$ is guaranteed, propagating recursively up and down the scale
hierarchy .

\subsection{Multi‑scale Monte‑Carlo experiment}

Our numerical protocol, adapted from
Refs.~\citep{turchin2003,newman2010}, tests whether approaching
$\mathcal{R}_{\text{crit}}$ anticipates social transitions:

\begin{description}
  \item[Scales] $S=5$ (family $\rightarrow$ civilisation).
  \item[Time step] $\Delta t = 1$ year; horizon 200 years.
  \item[Initial state] $P(s,0)\sim\mathcal{N}(0,1)$ (empirical calibration).
  \item[Dynamics]
    $\partial_{t}P = a\,\xi - b\,\Lambda$
    with pink noise $\xi$ ($1/f$ spectrum) and damping
    $\Lambda=\nabla^{2}_{s}P$.
\end{description}

Parameter values $a,b$ are chosen such that the power spectrum of
$P(s,t)$ matches historical log‑population series
(1800–2020) .

A total of 100 Monte‑Carlo realisations were executed; output variables
included $\mathcal{R}_{\max}$ and percentage of agents undergoing
state‑collapse (sign inversion).  All code and data are archived on
Zenodo (DOI forthcoming).

%-----------------------------------------------------------------
