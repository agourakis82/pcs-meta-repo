\section*{Appendix A — Glossary of Symbolic Constructs}
\addcontentsline{toc}{section}{Appendix A — Glossary of Symbolic Constructs}

\begin{description}
  \item[Symbolic Potential Field] A metaphorical construct representing the field of meaning, affect, and structure in which individuals are embedded; analogous to a gravitational or scalar field.

  \item[Social Hamiltonian] The symbolic equivalent of the physical Hamiltonian, representing the energy landscape of social configurations driven by symbolic, emotional, and cultural forces.

  \item[Wave Function Collapse of Meaning] The moment in symbolic cognition when a cloud of possible meanings resolves into a singular, communicable form — often triggered by attention, language, or emotional salience.

  \item[Antipair] The dialectical opposite of an idea, node, or symbolic configuration; essential for maintaining systemic tension and symbolic evolution.

  \item[Symbolic Mass] A theoretical resistance to change in an individual's symbolic or ideological position; the greater the symbolic mass, the less influence exerted by external symbolic fields.

  \item[Symbolic Tunnelling] The phenomenon by which a subject or structure transitions into a symbolically distant configuration without passing through intermediate logical states — inspired by quantum tunnelling.

  \item[Distributed Consciousness] A model of awareness not localised in individual brains but spread across symbolic structures, institutions, language systems, and artificial agents.

  \item[Fractal Recursion] The principle that symbolic, cultural, and cognitive forms reappear at multiple scales, with variation, echoing structural self-similarity.

  \item[Entropy of Signification] A measure of the multiplicity and indeterminacy of possible meanings within a symbol or message, which increases with abstraction or ambiguity.

  \item[Noospheric Layering] The stratification of consciousness into semiotic, linguistic, technological, and collective domains, inspired by Teilhard de Chardin's noosphere.
\end{description}
