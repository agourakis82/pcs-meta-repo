\section{Discussion}\label{sec:discussion}

The dual formulation—Bernoulli flow (H1) coupled to the fractal
continuity equation (H2)—reveals a phase portrait in which social
ruptures modulate, but seldom annihilate, a deep‑seated
entropy‑fractal attractor.  Three insights emerge.

\begin{enumerate}
\item \textbf{Robustness of $D_1$.}  Even with a rupture amplitude
      $\Lambda_0 = 0.8$, the network converges to
      $D_1^{\text{eq}} \simeq 1.17$.  This echoes meta‑analyses of human
      egonets \citep{zhou2005discrete} and resembles the
      core–periphery pattern reported in economic graphs
      \citep{borgatti2000models}.
\item \textbf{Recovery time‑scale.}  The latency
      $\Delta t \approx 6.5\%$ of the simulated horizon is comparable
      to the damping phase observed in post‑pandemic mobility traces
      \citep{gao2022collective}.
\item \textbf{Rupture as an entropic sink.}  The term
      $\Lambda \rho^\ast$ behaves as a negative source in
      Eq.~(\ref{eq:fract_continuity}), akin to a damping term in
      Fisher–KPP dynamics \citep{murray2002mathematical}, yet expressed
      on a Hausdorff manifold.
\end{enumerate}

\paragraph{Limitations.}
(i) Parameters $\alpha$ and $\beta$ were held constant; multiple shocks
may warrant time‑dependent $\alpha(t)$, $\beta(t)$.
(ii) The estimator $D_q$ is sensitive to degree binning; genuine
multiresolution methods \citep{mandelbrot1983fractal} could refine
confidence intervals.
(iii) Experiments were run for $N = 10^4$; ultra‑dense regimes remain to
be explored.

\paragraph{Outlook.}
Future work will transplant the formalism to hyperbolic spaces
\citep{krioukov2010hyperbolic} and fuse it with empirical blockchain
graphs, whose transactional geometry is inherently fractal.
