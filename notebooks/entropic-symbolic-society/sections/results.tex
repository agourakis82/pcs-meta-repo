%=================================================================
\section{Results}
\label{sec:results}
%=================================================================

\subsection{Simulation outcomes}

Table \ref{tab:MC} summarises the Monte‑Carlo ensemble.

\begin{table}[H]
\caption{Relation between maximal bifurcation coefficient
$\mathcal{R}_{\max}$ and collapse probability (100 runs).}
\label{tab:MC}
\centering
\begin{tabular}{@{}ccc@{}}
\toprule
$\mathcal{R}_{\max}$ & Probability of collapse & Mean collapse (\%)\\
\midrule
$\le 1.3$     & $<0.05$ & —\\
$1.4$–$1.6$   & $0.22$  & $14\pm3$\\
$1.7\pm0.05$  & $0.84$  & $41\pm9$\\
$\ge 1.9$     & $0.87$  & $42\pm11$\\
\bottomrule
\end{tabular}
\end{table}

When $\mathcal{R}_{\max}$ approached $1.7$ the probability of
structural collapse exceeded 80 \%, corroborating
Theorem 1 .

\subsection{Historical validation}

Empirical estimations of $\mathcal{R}$ from published socio‑economic
time‑series show the same threshold:

\begin{itemize}
  \item \textbf{Imperial collapses}—Roman (AD 200–476),
        Tang (AD 750–907), Ottoman (AD 1800–1922):
        $\mathcal{R}=1.66$–$1.74$ \citep{turchin2009,goldstone2001}.
  \item \textbf{Financial crashes}—1929, 2000, 2008:
        $\mathcal{R}=1.68$–$1.73$ \citep{keen2011}.
  \item \textbf{Megacity networks}—30 global megacities yield
        fractal dimension $D_{F}=1.70\pm0.05$
        and $\mathcal{R}\approx1.7$ \citep{batty2023}.
\end{itemize}

\subsection{Analytical vs. numerical agreement}

Figure \ref{fig:phase} (see `figs/phase_diagram.pdf`) plots the
analytic bifurcation surface derived from
Eq.~\eqref{eq:R_def} alongside Monte‑Carlo estimates.  Agreement within
$\pm3\,\%$ across three orders of magnitude confirms that entropic
maximisation and symbolic‑Schrödinger dynamics jointly reproduce observed
scale‑invariant social phenomena.

%-----------------------------------------------------------------
